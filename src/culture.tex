\section{A Harder Challenge is Changing Culture}

% These were just miscellaneous notes that I'd like to fit in, not sure how that looks yet

If DevOps is a culture and community that emphasizes collaboration between "operations" and "development," would RSE-ops be the same idea but focusing on the collaboration between HPC system administrators, and research software engineers? And instead of automating services, would we want to automate scientific workflows, only with small services when they are needed? Some might argue that it isn't about the tools per-say, but the practices and culture around them. So it simply is not enough to just create a new tool or state a practice without asking how it fits into current culture. From the other side of the coin, having an open mind is also important. If any member of an organization doesn't have a mindset that is open and amenable to change and innovation, it's going to be hard to practice. This might be especially challenging in the HPC/RSE community where stability and consistency (arguably the opposite of change) is already part of the culture.
A strategy to go about inspiring change might be to start with a current practice, and ask the questions:
 
 \begin{itemize}
\item "How do we automate this?"
 \item "How do teams X and Y work together, what are the common goals and units of operation?"
\end{itemize}

A lot of RSE-ops seems to come down to automation, which is why it's a good question to start with.
Another good suggestion is to consider RSE-ops from the level of the system. Each of these components can be discussed in terms of who works on it, how that is collaborative, and how it could be more collaborative or automated.

\begin{itemize}
\item booting
\item web servers and hosting static and dynamic sites
\item process management
\item ssh
\item file systems and volumes
\item system logging, monitoring, troubleshooting
\item protocols like SSL, TLS, TCP, UDP, SFTP, etc.
\item managing services (e.g., initd, systemd)
\item load balancing
\item breaking things and troubleshooting (this might be good for practice or learning to work together)
\end{itemize}

Another interesting question is how could this stack span HPC and cloud? How can we find some convergence? If we can create a map of cloud services, we could then try to map that to HPC. We ultimately want a layout of the land for what constitutes the "infrastructure" of HPC and what is missing or could be improved. Each node in this map would then be linked to documentation in a consistent, well-branded way.
