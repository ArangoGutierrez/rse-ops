\section{Mapping the Space}

During this exercise, we can see that there are two different communities that come together to form what we might call the HPC community, and you might guess this comes down to users (researchers, research software engineers) and admins (system administrators and also research software engineers). You'll notice that research software engineers can be part of both groups, which is why the profession has been helpful to shed light on the needs of the user base. The two groups include:

\begin{itemize}
\item users
\item admins
\end{itemize}

And the two groups can view the same ecosystem in very different ways. The two groups can also have some overlap in shared value for software or services, however not everything has perfect overlap. In many cases, even the user community is broken into smaller user communities based on domain of science, or preferred tools. For example, a physics group at a national lab might use a lot of MPI or Fortran codes, while a neuroimaging lab's bread and butter is launching SLURM jobs with a container technology.  This realization requires us to make a distinction between the two groups, and clarify that RSE-ops does not include domain or scientific libraries, but all of the testing, infrastructure, and support tools around them. We are making an effort to map out this space, which can be viewed at \url{https://vsoch.github.io/rse-ops/}. We are also clearly noting that RSE-ops extends beyond HPC (but includes it).
