\subsection{What is DevOps?}

A definition of Research Software Engineering Operations (RSE-ops) can best be derived by first explaining the philosophy behind DevOps \cite{aws} is a term that refers to best practices to bridge development and operations. It was coined in 2008 \cite{devops-paper}, and has grown out of a web services oriented model. The term has expanded to refer to other subsets of expertise in this area such as cloud security operations, which is called "DevSecOps," adding the "Sec" for "Security." In DevOps, best practices are generally defined around:

\begin{enumerate}
\item continuous integration: automated integration of code changes into a repository
\item continuous delivery: automated release of software in short cycles
\item monitoring and logging: tracking errors and system status in an automated way
\item communication and collaboration: interaction between team members and optimally working together
\item "infrastructure as code": provisioning resources through simple text files
\item using micro-services: single function modules that can be assembled into a more complex service
\end{enumerate}


The above best practices are done for the purposes of speed and efficiency, reliability, scale, and collaboration. It has been shown that teams that adopt these practices can see improvements in productivity, efficiency, and quality across the board \cite{devops-changed-things}. It is a culture because along with these best practices, it also alludes to a way of thinking and working. Where there were barriers before between development and operations teams, DevOps brought them down. You can grow a community around these ideas, which is a powerful thing. 

\subsubsection{DevOps as the Driver of the Cloud}

And surely the statistics are alarmingly good, as teams that practice DevOps outperform their peers in number and speed of deployments, recovery from downtime events, and employee  ability to work on new things over tenuous maintenance \cite{devops-stats-2016}. Recognizing these gains and providing structure for collaboration, training, and projects was arguably just one of the goals of the Cloud Native Computing Foundation (CNCF), which was founded in 2015 \cite{Wikipedia_contributors2021-jt}. Specifically, the primary stated reason for foundation of CNCF was to foster community and support around container technologies, which often are the core unit of automation and DevOps practices \cite{Nishanil_undated-fk}). A new term, "cloud-native" was coined with this title, which is heavily reliant on DevOps. DevOps practices are considered the fundamental base of taking on a cloud-native approach, and another term, "Cloud Native DevOps" \cite{cloud-native-devops-oreilly} was even coined to specifically refer to the application of DevOps practices to the cloud. Since the two are so inexplicably intertwined, for the remainder of this paper, we will refer to them interchangeably \cite{Choice2020-lt}.

% TODO: make a diagram or table explaining DevOps, RSE-ops, etc.
% TODO: how to link site with content here, and provide list of software for each?

